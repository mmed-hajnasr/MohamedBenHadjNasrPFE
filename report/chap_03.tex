\chapter{Analatycs and Requirements specifications}

\section*{Introduction}
This chapter meticulously examines the current environment, including its existing architecture and deployment process. This in-depth analysis will ensure comprehensive coverage of both functional and non-functional needs and requirements. We'll then establish a roadmap for utilizing the Scrum process, defining objectives and deliverables along the way.

\section{Infrastructure Study}
The company is currently using the baseline web application architecture provided by Microsoft\cite{webArticle6}.
This figure \ref{fig:gloabal_architecture} presents the global architecture of the application.

\begin{figure}[htpb]
    \centering
    \frame{\includegraphics[width=0.8\columnwidth]{gloabal_infrastructure.png}}
    \caption{gloabal architecture}
    \label{fig:gloabal_architecture}
\end{figure}

\pagebreak

\subsection{Description:} The architecture exposes a public endpoint via Azure Application Gateway with a Web Application Firewall. The App Service application uses virtual network integration to securely communicate to Azure PaaS services such as Azure Key Vault and Azure SQL Database.

\subsection{Componants of the architecture}
\begin{itemize}
    \item \textbf{Virtual Network:} This is the fundamental building block for your private network in Azure. It provides isolation and protection for your resources.
    \item \textbf{App Service:} This service is used to host the web application. It provides a fully managed platform for building, deploying, and scaling web apps.
    \item \textbf{Azure SQL Database:} This service is used to store the application data. It provides a fully managed relational database with built-in high availability and security.
    \item \textbf{Azure Key Vault:} This service is used to store and manage application secrets. It provides a secure and centralized storage for application secrets.
    \item \textbf{Azure Application Gateway:} This service is used to protect the web application from common web vulnerabilities. It provides a web application firewall and other security features.
    \item \textbf{Azure Monitor:} This service is used to monitor the health of the web application. It provides logging and application telemetry to monitor the health of the application.
    \item \textbf{Azure DevOps:} This service is used to automate the deployment of the web application. It provides a set of tools for building, testing, and deploying applications.
    \item \textbf{Virtual Interface:} This service is used to connect the web application to the virtual network. It provides a secure and private connection to the web application.
    \item \textbf{Application Insights:} This service is used to monitor the performance of the web application. It provides real-time monitoring and analytics for the web application.
    \item \textbf{Private DNS Service:} This service is used to resolve the DNS names of the Azure PaaS services. It provides a secure and private DNS resolution for the web application.
    \item \textbf{Private endpoint:} This service is used to connect the web application to the Azure PaaS services. It provides a secure and private connection to the Azure PaaS services.
\end{itemize}

\subsection{Network flows}
\textbf{Inbound flow:}
\begin{itemize}
    \item The user issues a request to the Application Gateway public IP.
    \item The WAF rules are evaluated.
    \item The request is routed to an App Service instance through the private endpoint.
\end{itemize}
\textbf{App Service to Azure PaaS services flow:}
\begin{itemize}
    \item App Service makes a request to the DNS name of the required Azure service. The request could be to Azure Key Vault to get a secret, Azure SQL Database.
    \item The request is routed to the service through the private endpoint.
\end{itemize}

\textbf{Deploying to the app service:} \\
The deployment process is initiated from the Azure portal by the admin from his machine.

\subsection{Architecture characteristics}
\begin{itemize}
    \item For security reasons, the network in this architecture has separate subnets for the Application Gateway, App Service integration components, and private endpoints. Each subnet has a network security group that limits both inbound and outbound traffic for those subnets to just what is required.
    \item The App Service baseline configures authentication and authorization for user identities (users) and workload identities (Azure resources) and implements the principle of least privilege.
    \item Azure Monitor collects and analyzes metrics and logs from your application code, infrastructure (runtime), and the platform (Azure resources).
\end{itemize}
\section{Deployment Process}
The process of releasing a new application version or update in our system follows a multi-step approach. This approach ensures a smooth transition from development to production and minimizes the risk of introducing issues.
Here's a breakdown of the key stages involved:
\begin{itemize}
    \item \textbf{Code Preparation:} During this stage, developers finalize the code for the new application version or update. This may involve tasks like code reviews, bug fixing, and integration with existing systems.
    \item \textbf{Testing:} Once the code is prepared, it undergoes rigorous testing by the Quality Assurance (QA) team. This testing verifies that the application functions as intended identifies and resolves any bugs or errors, and ensures compatibility with different environments.
    \item \textbf{Deployment Scheduling:} Following successful testing, a deployment plan is created. This plan defines the specific time and method for releasing the application to users. The plan often involves coordination between development, QA, and system administration teams to ensure a smooth rollout and minimal disruption to ongoing operations.
    \item \textbf{Deployment:} During deployment, the application is transferred from its development environment to the production environment where it will be used by end-users. This process may involve tasks like uploading application files, configuring settings, and integrating with databases or other systems.
    \item \textbf{Monitoring:} After deployment, the system administrators closely monitor the application's performance and functionality. This monitoring helps to identify any issues that may arise after the release and allows for prompt intervention if necessary.
\end{itemize}
Effective collaboration between development, QA, and system administration teams is crucial throughout this process. Clear communication and well-defined roles ensure a successful application deployment with minimal downtime and a positive experience for end-users.

\section{Needs and requirements}
To be able to improve the current deployment process while ensure the security and performance of our applications, we have identified the following needs and requirements:
\noindent
\subsection{Functional Needs:}
\begin{itemize}
    \item \textbf{Patch Deployment Automation:} Implement a system for automated deployment of patches to applications and systems.
    \item \textbf{Version Rollback Capability:} Provide the ability to revert to a previous version of an application during deployment in case of failure or unexpected issues.
\end{itemize}
\subsection{Non-functional Needs:}
\noindent
\begin{itemize}
    \item \textbf{Availability:} Ensure high availability of applications and services, minimizing downtime during deployments.
    \item \textbf{Security:} Implement Web Application Firewall (WAF)) to safeguard applications from cyber threats.
    \item \textbf{Performance:} Optimize deployment processes to maintain optimal performance levels of applications and systems.
    \item \textbf{Cost optimization:} Minimize the costs associated with deployment processes, including recourses and infrastructure.
\end{itemize}

\subsection{Objectives}
By analysing the needs and requirements, we have defined the following objectives and deliverables that will be provided during multiple sprints:
\begin{longtable}[c]{
    |p{.20\textwidth}
    |p{.55\textwidth}|
    p{.21\textwidth}|
    }
    \caption{the key technical objectives for the project}
    \label{tab:objectivesTable}                      \\
    \hline

    the Sprints
     & Objectives
     & Delivrables                                   \\
    \hline

    Sprint1(2 weeks)
     & Probision the cloud infrastructure as IaC
     & Terraform scripts                             \\
    \hline

    Sprint2(2 weeks)
     & Implement the Continuous Integration pipeline
     & a DevOps workflow                             \\
    \hline

    Sprint3(2 weeks)
     & Set up a deployment strategy
     & seamless deployment                           \\
    \hline
\end{longtable}

\begin{figure}
    \centering
    \frame{\includegraphics[width=0.84\columnwidth]{grantt.png}}
    \caption{This is the grantt chart of the scrum process}
    \label{fig:grantt_chart}
\end{figure}

\section*{Conclusion}
This chapter's comprehensive analysis of our cloud environment, infrastructure, and deployment process provides a firm foundation for the next crucial step: constructing a secure, efficient, and automated cloud-based deployment framework. With this groundwork laid, we can confidently transition to the realization phase, detailed in the next chapter, where we'll navigate the provisioning of our current infrastructure.
