\chapter*{General Introduction}
\addcontentsline{toc}{chapter}{General Introduction}
\markboth{General Introduction}{}
\noindent In the fast-paced world of web development, users expect applications and websites to be constantly available and up-to-date. This creates a crucial challenge: how to implement new features, fix bugs, and improve performance without disrupting the user experience.
\par
\noindent Furthermore, to guarantee an impeccable user experience, testing is a vital step in the development process. However, this testing needs to be streamlined to avoid slowing down development.
\par
\noindent This need for efficiency necessitates an automated deployment process. For cloud-hosted websites, setting up such a process presents its own challenges, such as connecting various cloud services and development tools. However, it also offers unique opportunities, like streamlining the creation of the cloud infrastructure itself.
\par 
\noindent ADACTIM, a cloud and outsourcing services company whose introduction can be found in the \textbf{first chapter}, aims to establish an optimized deployment and testing process for a critical cloud-hosted web application. This will accelerate development and allow developers to focus on core coding rather than deployment tasks.
\par 
\noindent The technical development will follow the DevOps process, requiring an understanding of cloud computing and DevOps practices that will be detailed in the \textbf{second chapter}.
\par
\noindent After that we will explore the thought process behind selecting the process components in the \textbf{third chapter}.
\par 
\noindent The realiasation of the project is detailed in the last three chapters. First in the \textbf{fourth chapter}, we'll establish a solid foundation for our cloud environment using Infrastructure as Code (IaC). This involves creating scripts that automate the provisioning of cloud resources. These scripts ensure our infrastructure is not only built efficiently but also remains reusable and maintainable for future changes.
\par
\noindent Next in the \textbf{fifth chapter}, we'll implement the (CI/CD) pipeline. This pipeline acts like an automated assembly line, taking code changes and automatically triggering builds, tests, and potentially even deployments. all while establishing a DevOps workflow will bridge the gap between development and operations teams, fostering collaboration and a smoother development process.
\par
\noindent Finally in the \textbf{sixth chapter}, we'll focus on creating deployment scripts and the pipeline flow that guarantees minimal downtime during updates to the live application.